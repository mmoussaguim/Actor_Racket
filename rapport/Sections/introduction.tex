\section{Introduction}

Le sujet de programmation  informatique traité ce semestre est le modèle basé sur les acteurs (\textbf{Actor Model}). C'est un modèle mathématique qui considère un monde remplie d'acteurs qui interagissent entre eux. Dans ce modèle un acteur est une unité fondamentale de calcul; c'est un objet qui reçoit un message et fait une sorte de calcul sur lui même. Tout acteur posséde alors un état propre, envoie mais aussi reçoit des messages d'autres acteurs et met à jour son propre état. \\
Le jeu \textit{Terminal Phase} codé par Christopher Lemmer Webber en langage Racket, est un exemple concret de ce type de programmation.
A l'instar de ce dernier, le but de notre projet est de construire et développer une bibliothéque d'acteurs en Racket menant à l'élaboration d'un jeu qui approche un shoot'em up. 